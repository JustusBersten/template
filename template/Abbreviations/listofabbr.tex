%!TEX root = ../main.tex

\chapter*{Abkürzungs- und Symbolverzeichnis}
\addcontentsline{toc}{chapter}{Abkürzungs- und Symbolverzeichnis}

\subsection*{Symbole}

Hier können Informationen zu den verwendeten Symbolen eingefügt werden.

\begin{symbollist}
	% Optional item argument makes the symbol/abbr
	\item[$\mathbb{X}$] Ein $X$ in \textit{blackboard bold}. Wie nützlich.
	\item[$\mathcal{X}$] Ein kalligrafisches $X$. Auch süß.
	\item[$\mathbf{X}$] Ein fettes $X$. Weniger süß.
	\item[$\mathfrak{X}$] $X$ in germanischer Fraktur geschrieben. Naja, wem es gefällt.
\end{symbollist}


\subsection*{Abkürzungen}

Hier sollen die verwendeten Abkürzungen eingefügt werden.

\begin{symbollist}
	\item[AD] Anno Domini, lateinisch für \textit{im Jahre des Herren}.
	\item[CAPM] Englisch für \textit{Capital Asset Pricing Model} - Ein Preismodell für Kapitalmarktgüter. Entwickelt in den 1960er und 70er Jahren von Sharpe (1964), Lintner (1965), Black (1972) basierend auf der Portfoliotheorie von Harry Markowitz (1952).
	\item[WDFNMDFSSGENEU] \textit{Während der Fahrt nicht mit dem Fahrer sprechen, sonst gibt es noch ein Unglück} - Die Länge der Abkürzung bzw. ist beschränkt auf etwa ein Viertel der Textbreite. Vorsicht, sonst leidet die Ästhetik.
\end{symbollist}
