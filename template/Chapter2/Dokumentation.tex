%!TEX root = ../main.tex

\chapter{Gebrauchsanweisung \& Dokumentation}
\label{chap:dokumentation}

Diese \LaTeX-Vorlage dient der Erstellung von Bachelor- und Masterarbeiten. Sie erfüllt die Richtlinien und vorausgesetzten Anforderungen an wissenschaftliche Arbeiten. Geschrieben und kompiliert wurde diese Vorlage von Marius Theiß im Frühjahr 2016. Sie basiert auf der 2004 \LaTeX2$\epsilon$ Version von scrreprt.cls, der KOMA-Script Version 3 report Klasse. Weitere Informationen und eine generelle Dokumentation zum KOMA-Script sind auf \url{http://www.komascript.de/} verfügbar.


Diese Datei ist zur Modifikation und Distribution freigegeben unter der \LaTeX-Project Public License Version 1.3c, oder jeder späteren Version. Die aktuelle Version ist zu finden unter \url{http://www.latex-project.org/lppl.txt}.

Bei dieser Vorlage handelt es sich um die Version vom 06. Juni 2016 von Marius Theiß\footnote{Kontakt E-Mail: \href{mailto:marius.theiss@udo.edu}{marius.theiss@udo.edu}}.


\section{Features}
\label{sec:features}

Diese Vorlage erfüllt die stilistischen Vorgaben an eine wissenschaftliche Bachelor- beziehungsweise Masterarbeit:
\begin{enumerate}
\item Zentrierte Seitenzahlen auf der Fußzeile jeder Seite
\item Seitennummerierung für die Verzeichnisse römisch, ab dem Volltext wird arabisch nummeriert, wie erforderlich.
\item Der Text ist als Blocksatz formatiert mit einer Serifenschrift in Schriftgröße 12pt
\item Formatierte Titelseite, Inhalts- sowie Literaturverzeichnis
\item Optional: Verzeichnisse für Abbildungen, Tabellen und ein Abkürzungs- und Symbolverzeichnis und Anhang
\item Optimiert für DIN A4 mit 1.5pt Zeilenabstand
\end{enumerate}


\section{Klassenoptionen}
\label{sec:optionen}

Das Herz dieser Vorlage ist die Datei \texttt{header.tex}. Dort wird die KOMA-Klasse \texttt{scrreprt} initialisiert, alle nötigen Umgebungen definiert, Pakete eingebunden und wichtigen Einstellungen vorgenommen.

Die Optionen für die Initialisierung der KOMA-Klasse \texttt{scrreprt} sollten nach Möglichkeit nicht verändert werden. Dennoch gibt es einige optionale Einstellungen, die bei Bedarf verändert werden können:

\begin{symbollist}
	\item[\texttt{chapterprefix}] Alle Kapitel welche mit {\verb \chapter{title} } initialisiert werden, beginnen mit dem Prefix \textit{Kapitel} und der zugehörigen Nummer.
	\item[\texttt{numbers}] Bestimmt ob die Prefixe der Kapitel mit einem Punkt beendet werden sollen. \textit{noenddot} ist hier die Standardeinstellung.
	\item[\texttt{parskip}] Paragraphen werden mit einer Leerzeile getrennt. Die Einstellung \textit{false} führt dazu, dass Paragraphen lediglich eingerückt werden.
	\item[draft] Optional: Grafiken werden nur als schwarze Umrandungen dargestellt. Optimal geeignet zum Drucken einer Vorabversion.
\end{symbollist}

Im Rest der Datei \texttt{header.tex} werden die Pakete geladen und benötigte Umgebungen definiert. Die Datei ist vollständig (auf englisch) kommentiert und lässt Platz für weitere Pakete und eigene Definitionen.

\section{Vorgehen}
Wer mit dieser Vorlage arbeiten will, dem wird das folgende Vorgehen empfohlen:

\begin{enumerate}
\item Im Pfad \texttt{./Titlepage/} die Titelseite ausfüllen
\item \texttt{header.tex} durchsehen, optionale Klassenoptionen auf Wunsch verändern. Weitere Pakete einbinden und Umgebungen definieren falls nötig.
\item \texttt{main.tex} durcharbeiten. Sind die ersten zwei Schritte erledigt, spielt sich alles rund um die Arbeit in \texttt{main.tex} ab.
\item Das Einbinden von Bildern findet ganz klassich mit { \verb \includegraphics{name} } statt. Die Bilder können in den eingestellten Standardpfad \texttt{./Pictures/} abgelegt werden und sind dann nur durch Einbinden des Dateinamens hinzuzufügen. Das Ausschreiben des Dateipfads ist damit redundant!
\end{enumerate}
