\documentclass[12pt,paper=a4,numbers=noenddot,toc=bibliography,toc=listof,chapterprefix,parskip=true]{scrreprt}
%-----------------------------------------------------------------------------%
% Default options: 
%	*	Optimized for DIN A4 output with font size 12. toc=bibliography adds 
%		the bibliography as a not numbered entry to the Table of Contents.
%		chapterprefix will print "Chapter #" on top of the page. parskip=true
%		will add one line vertical space between paragraphs. See the
%		KOMA-Script documentation for more options.
%
% Other useful options:
%	*	draft	--	don't actually include images, print a black bar on 
%					overful hboxes.
%	*	
%
%-----------------------------------------------------------------------------%


% Useful packages for academic writing:
\usepackage{amsmath, amssymb, amsfonts}
\usepackage{stmaryrd}
\usepackage{graphicx}
\usepackage{natbib}
\usepackage{color}
\usepackage{bm}
\usepackage{subcaption}
\usepackage{graphicx}
\usepackage{mathabx}
\usepackage{array}
\usepackage{booktabs}
\usepackage{tabularx}
\usepackage{longtable}
\usepackage{multirow}
\usepackage{pbox}
%\usepackage{cite}	% Not recommended: Can lead to break of hyperlink cites
					% Only use this if you want your bibliography to sort
					% entries by order-of-use.

% Personal commands and abbreviations:
%Define and personal commands here


%Graphics Path to find your pictures:
\graphicspath{{./Pictures/}}


%-----------------------------------------------------------------------------%
% LANGUAGE: Define preferred language (Default: German). 
%-----------------------------------------------------------------------------%
\usepackage[utf8]{inputenc}
\usepackage[TS1,T1]{fontenc}
\usepackage{lmodern,textcomp}
\renewcommand\abstractname{Abstract}
%Change language settings here:
\usepackage[english,ngerman]{babel} % Sets language to German (Default)
 \shorthandoff{""}
%\usepackage[ngerman, english]{babel} % Sets language to English


%-----------------------------------------------------------------------------%
% CAPTION: Define preferred caption style. 
%-----------------------------------------------------------------------------%
\usepackage[format=plain,labelfont=sc]{caption} 
%-----------------------------------------------------------------------------%


%-----------------------------------------------------------------------------%
% HYPERREF: plain black hypertext references for ref's and cite's.
%-----------------------------------------------------------------------------%
\usepackage[pdftex, pdfusetitle, plainpages=false,
			bookmarks, bookmarksnumbered,
			colorlinks, linkcolor=black, citecolor=black,
		    filecolor=black, urlcolor=black]{hyperref}
%-----------------------------------------------------------------------------%


%-----------------------------------------------------------------------------%
% PAGESTYLE: Define Formatting of page headers & footers and sets 1.5 spacing. 
%-----------------------------------------------------------------------------%
\usepackage{scrpage2}
\usepackage[onehalfspacing]{setspace}
\recalctypearea
\pagestyle{scrheadings}
\automark[section]{chapter}
\addtokomafont{sectioning}{\rmfamily}

\addtokomafont{paragraph}{\itshape} % paragraph italics
\addtokomafont{subparagraph}{\itshape}
%-----------------------------------------------------------------------------%


%-----------------------------------------------------------------------------%
% THEOREM: Define theorem style and formats. 
%-----------------------------------------------------------------------------%
%theorem environment for german language settings (default)
\usepackage{ntheorem}
\theoremstyle{break}
\theoremheaderfont{\sffamily\bfseries}
\theorembodyfont{\upshape}
\theoremsymbol{}
\newtheorem{definition}{Definition}[chapter]
\newtheorem{satz}[definition]{Satz}
\newtheorem{resultat}[definition]{Resultat}
\newtheorem{lemma}[definition]{Lemma}
\newtheorem{folgerung}[definition]{Folgerung}
\newtheorem{korollar}[definition]{Korollar}
\theorembodyfont{\rmfamily}
\newtheorem{bemerkung}[definition]{Bemerkung}
\newtheorem{beispiel}[definition]{Beispiel}
% style of proof environment
\theoremstyle{nonumberplain}
\theoremsymbol{\ensuremath{\Box}}
\newtheorem{beweis}{Beweis:}


%%Uncomment for english language settings
%\usepackage{amsthm}
%\newtheorem{theorem}{Theorem}%[section]
%\newtheorem{lemma}[theorem]{Lemma}
%\newtheorem{proposition}[theorem]{Proposition}
%\newtheorem{corollary}[theorem]{Corollary}
%\newtheorem{result}[theorem]{Result}
%-----------------------------------------------------------------------------%



%==============================================================================
% DO NOT CHANGE ANYTHING BEYOND THIS POINT
%-----------------------------------------------------------------------------%
% DEVELOPER: Necessary definitions and packages for debugging
%-----------------------------------------------------------------------------%
\usepackage{scrhack} % Fix packages only partial compatible with KOMA-script
% New environment for list of symbols
\newenvironment{symbollist}
{%
\normalbaselines
\begin{list}{ }{%
	\addtolength{\topsep}{5.0pt}
	\addtolength{\itemsep}{5.0pt}
	\setlength{\labelwidth}{0.24\textwidth}
	\setlength{\labelsep}{2em}
	\setlength{\parsep}{0pt}
	\setlength{\leftmargin}{\labelwidth}
	\setlength{\rightmargin}{0pt}
	}
}
{ \end{list} } 
